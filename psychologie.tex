% !TeX encoding = UTF-8
% !TeX spellcheck = de_DE
\documentclass[12pt]{scrartcl}

\usepackage[utf8]{inputenc}
\usepackage[T1]{fontenc}
\usepackage[ngerman]{babel}
\usepackage{xspace}
\DeclareRobustCommand{\zB}{z.\,B.\xspace}

\begin{document}
\tableofcontents \pagebreak

\section{Gruppe}

\subsection{Einführung und Überblick}
Die \emph{Gruppenpsychologie} (Teil der \emph{Sozialpsychologie}) beschäftigt
sich mit den \emph{Auswirkungen} der \emph{sozialen Umwelt} auf den
\emph{einzelnen Menschen}.

\paragraph{zwei Teildisziplinen:}
\begin{enumerate}
	\item Gruppenforschung (empirisch/experimentell arbeitend)
	\item angewandte Gruppendynamik
\end{enumerate}

\paragraph{Ziel:} überdauernde Verhaltensänderungen

\subsubsection*{zu 1:} seit Mitte des letzten Jahrhunderts (Lewin: 1890-1947),
Höhepunkt um und nach dem 2. Weltkrieg (zwischen 1950 und 1975 wichtigster
Zweig der Sozialpsychologie)

\paragraph{Themen:}
\begin{itemize}
	\item Gruppendynamik
	\item Gruppenbildung/Motive
	\item Gruppenzusammenhalt
	\item Gruppenspezifische Verhaltensweisen (Zwänge?)
	\item Rangordnung
	\item Kommunikation
	\item Vor- und Nachteile
	\item Kooperation
	\item Entstehen von Einstellungen/Urteilsvermögen/Wahrnehmung
\end{itemize}

\paragraph{Ziel:} Verhaltensänderungen (Vermeidung politischer Katastrophen: J.
L. Moreno, 1966)

\subsubsection*{zu 2:} \emph{Gruppendynamische Experimente}

\subsection{Definition und Arten}

\subsubsection{Gruppenmerkmale} \label{subsec:Gruppenmerkmale}
Für den sozialpsychologischen Begriff der Gruppe sind die wechselseitigen
Erwartungen der Mitglieder untereinander konstituierend: Als Mitglied der
Gruppe glaube ich sicher zu sein, wie die anderen Mitglieder mich sehen; ich
selbst habe mir eine genaue Vorstellung aller anderen erworben; gleichzeitig
weiß ich aber auch, dass alle anderen über diese Dinge Bescheid wissen oder
wenigstens feste Vermutungen darüber haben. So wird das Verhalten der
Individuen in der Gruppe von solchen Hypothesen auf verschiedenen Ebenen
gesteuert. Neben dem genannten Hauptmerkmal der Interdependenz, durch das
gleichzeitig die Zahl der Mitglieder auf 8 bis 12 beschränkt ist, gibt es eine
Reihe weiterer Merkmale, die für eine sozialpsychologische Gruppe von Bedeutung
sind:
\begin{itemize}
	\item Interaktionen und gefühlsmäßige Wechselbeziehungen aller Mitglieder
		untereinander
	\item Strukturierung durch allgemein anerkannte Positionen und Rollen zur
		Organisation von Abläufen innerhalb der Gruppe
	\item Gruppenziel und gemeinsame Normen auf gruppenrelevantem Gebiet
	\item Zusammengehörigkeitsgefühl im Sinne von Gruppenbewusstsein als
		Identifikation der Mitglieder mit der Gruppe und als einheitliches
		Verhalten gegenüber der Außenwelt
	\item Relative Dauer der Gruppenzugehörigkeit und Erleben von Kontinuität
		durch Traditionen und Gewohnheiten.
\end{itemize}
Kein brauchbares Merkmal der sozialpsychologischen Gruppe ist die sogenannte
„Gruppenseele“, auch wenn man häufig feststellen kann, dass Gruppen ihren
Charakter über lange Zeit und trotz Mitgliederwechsel bewahren. Damit würde der
Gruppe eine Eigenschaft zugeschrieben, die sie nicht haben kann, weil sie als
Person nicht existiert, sondern lediglich als Summe ihrer Mitglieder. Diese
bewirken durch ihr aufeinander bezogenes Verhalten in Vorstellungen,
Hoffnungen, Konflikten einen dynamischen, affektiv und kognitiv gesteuerten
Prozess, der vom Außenstehenden ganzheitlich wahrgenommen und als typisch für
diese Gruppe gesehen wird (\zB Mitglieder einer Jugendbande, Schüler einer
bestimmten Klasse).

\subsubsection{Gruppeneinteilung}
Der Versuch, die vielen sozialpsychologischen Gruppen systematisch einzuteilen,
ist schwierig, weil sich nur selten eindeutige Zuordnungen treffen lassen. In
der Regel unterscheidet man zunächst die Primärgruppen (Familie, Großfamilie,
Nachbarschaft), zu denen jeder seit seiner Kindheit gehört und denen er
lebenslang verhaftet bleibt, von Sekundärgruppen, denen man sich erst später
und häufig auch nur vorübergehend anschließt (Schulkameraden, Kollegen).
Sekundärgruppen werden ihrerseits in formelle und informelle Gruppen
aufgeteilt. Formelle Gruppen entstehen durch äußere Bedingungen: Eine
Schulklasse wird neu gebildet, im Studium arbeiten verschiedene Studenten im
gleichen Labor, eine Betriebsabteilung wird geschaffen. Es gibt keine formelle
Gruppe, die nicht im Lauf der Zeit zur informellen oder zu mehreren informellen
würde. Die informelle Gruppe (Clique innerhalb der Klasse, Freundeskreis)
beruht vor allem auf den gefühlsmäßigen Beziehungen der Mitglieder
untereinander. Besteht die informelle Gruppe aus nur zwei Mitgliedern, so nennt
man sie Dyade (Ehe, Zweierbeziehung, Freundespaar). In ihr werden die
wechselseitigen Erwartungen und Hoffnungen, die Abhängigkeit des einen von
Entscheidungen, von Gefühlen des anderen besonders deutlich. Wird die eigene
Gruppe von dem Mitgliedern selbst positiv wahrgenommen, so spricht man von
einer Wir-Gruppe (Ingroup), wogegen die anderen, die der eigenen Gruppe nicht
angehören, als Die-Gruppe oder Outgroup eher negativ bewertet werden.
Schließlich spricht man von einer Bezugsgruppe, wenn ein Individuum sich an den
Eigenarten und Erwartungen dieser Gruppe misst.

\subsubsection{Abgrenzungen}
\begin{itemize}
	\item[Menge:] (\emph{Def.:}) Anzahl von Menschen, die zufällig zu
		einem bestimmten Zeitpunkt zusammen sind (instabil), \zB in
		einer Bahnhofshalle, im Supermarkt, ...

		Eine Menge kann werden zur:
		\begin{itemize}
			\item[Masse:] (\emph{Def.:}) große Zahl von Menschen auf
				engem Raum, einheitlich auf ein Ziel ausgerichtet
				(gefühlsmäßig); Massenphänomen: \emph{Panik} (Binnenstruktur
				einer Gruppe fehlt)
			\item[psycholog. Gruppe:] siehe Abschnitt
				\ref{subsec:Gruppenmerkmale}
		\end{itemize}
	\item[Klasse:] (\emph{Def.:})
		\begin{itemize}
			\item Anzahl von Menschen, die durch ein \emph{bestimmtes Merkmal}
				gekennzeichnet sind, \zB
				\begin{itemize}
					\item Frauen
					\item Männer
					\item Mercedesfahrer
				\end{itemize}
			\item Zuweisung zu einer \emph{sozialen Schicht:}
				\begin{itemize}
					\item Einkommen
					\item Wohngegend
					\item Herkunft, Hautfarbe
					\item Schulbildung
					\item Berufsausbildung
				\end{itemize}
		\end{itemize}
		\item[Verband:]
			\begin{itemize}
				\item Gewerkschaften
				\item Fußballverein
			\end{itemize}
			Charakteristika:
				\begin{itemize}
					\item gemeinsames Ziel
					\item offizieller Charakter
				\end{itemize}
\end{itemize}

\end{document}
