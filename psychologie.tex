% !TeX encoding = UTF-8
% !TeX spellcheck = de_DE
\documentclass[12pt]{scrartcl}

\usepackage[utf8]{inputenc}
\usepackage[T1]{fontenc}
\usepackage[ngerman]{babel}
\usepackage{xspace}
\DeclareRobustCommand{\zB}{z.\,B.\xspace}

\begin{document}
\tableofcontents \pagebreak

\section{Gruppe}

\subsection{Einführung und Überblick}
Die \emph{Gruppenpsychologie} (Teil der \emph{Sozialpsychologie}) beschäftigt
sich mit den \emph{Auswirkungen} der \emph{sozialen Umwelt} auf den
\emph{einzelnen Menschen}.

\paragraph{zwei Teildisziplinen:}
\begin{enumerate}
	\item Gruppenforschung (empirisch/experimentell arbeitend)
	\item angewandte Gruppendynamik
\end{enumerate}

\paragraph{Ziel:} überdauernde Verhaltensänderungen

\subsubsection*{zu 1:} seit Mitte des letzten Jahrhunderts (Lewin: 1890-1947),
Höhepunkt um und nach dem 2. Weltkrieg (zwischen 1950 und 1975 wichtigster
Zweig der Sozialpsychologie)

\paragraph{Themen:}
\begin{itemize}
	\item Gruppendynamik
	\item Gruppenbildung/Motive
	\item Gruppenzusammenhalt
	\item Gruppenspezifische Verhaltensweisen (Zwänge?)
	\item Rangordnung
	\item Kommunikation
	\item Vor- und Nachteile
	\item Kooperation
	\item Entstehen von Einstellungen/Urteilsvermögen/Wahrnehmung
\end{itemize}

\paragraph{Ziel:} Verhaltensänderungen (Vermeidung politischer Katastrophen: J.
L. Moreno, 1966)

\subsubsection*{zu 2:} \emph{Gruppendynamische Experimente}

\subsection{Definition und Arten}

\subsubsection{Gruppenmerkmale} \label{subsec:Gruppenmerkmale}
Für den sozialpsychologischen Begriff der Gruppe sind die wechselseitigen
Erwartungen der Mitglieder untereinander konstituierend: Als Mitglied der
Gruppe glaube ich sicher zu sein, wie die anderen Mitglieder mich sehen; ich
selbst habe mir eine genaue Vorstellung aller anderen erworben; gleichzeitig
weiß ich aber auch, dass alle anderen über diese Dinge Bescheid wissen oder
wenigstens feste Vermutungen darüber haben. So wird das Verhalten der
Individuen in der Gruppe von solchen Hypothesen auf verschiedenen Ebenen
gesteuert. Neben dem genannten Hauptmerkmal der Interdependenz, durch das
gleichzeitig die Zahl der Mitglieder auf 8 bis 12 beschränkt ist, gibt es eine
Reihe weiterer Merkmale, die für eine sozialpsychologische Gruppe von Bedeutung
sind:
\begin{itemize}
	\item Interaktionen und gefühlsmäßige Wechselbeziehungen aller Mitglieder
		untereinander
	\item Strukturierung durch allgemein anerkannte Positionen und Rollen zur
		Organisation von Abläufen innerhalb der Gruppe
	\item Gruppenziel und gemeinsame Normen auf gruppenrelevantem Gebiet
	\item Zusammengehörigkeitsgefühl im Sinne von Gruppenbewusstsein als
		Identifikation der Mitglieder mit der Gruppe und als einheitliches
		Verhalten gegenüber der Außenwelt
	\item Relative Dauer der Gruppenzugehörigkeit und Erleben von Kontinuität
		durch Traditionen und Gewohnheiten.
\end{itemize}
Kein brauchbares Merkmal der sozialpsychologischen Gruppe ist die sogenannte
„Gruppenseele“, auch wenn man häufig feststellen kann, dass Gruppen ihren
Charakter über lange Zeit und trotz Mitgliederwechsel bewahren. Damit würde der
Gruppe eine Eigenschaft zugeschrieben, die sie nicht haben kann, weil sie als
Person nicht existiert, sondern lediglich als Summe ihrer Mitglieder. Diese
bewirken durch ihr aufeinander bezogenes Verhalten in Vorstellungen,
Hoffnungen, Konflikten einen dynamischen, affektiv und kognitiv gesteuerten
Prozess, der vom Außenstehenden ganzheitlich wahrgenommen und als typisch für
diese Gruppe gesehen wird (\zB Mitglieder einer Jugendbande, Schüler einer
bestimmten Klasse).

\subsubsection{Gruppeneinteilung}
Der Versuch, die vielen sozialpsychologischen Gruppen systematisch einzuteilen,
ist schwierig, weil sich nur selten eindeutige Zuordnungen treffen lassen. In
der Regel unterscheidet man zunächst die Primärgruppen (Familie, Großfamilie,
Nachbarschaft), zu denen jeder seit seiner Kindheit gehört und denen er
lebenslang verhaftet bleibt, von Sekundärgruppen, denen man sich erst später
und häufig auch nur vorübergehend anschließt (Schulkameraden, Kollegen).
Sekundärgruppen werden ihrerseits in formelle und informelle Gruppen
aufgeteilt. Formelle Gruppen entstehen durch äußere Bedingungen: Eine
Schulklasse wird neu gebildet, im Studium arbeiten verschiedene Studenten im
gleichen Labor, eine Betriebsabteilung wird geschaffen. Es gibt keine formelle
Gruppe, die nicht im Lauf der Zeit zur informellen oder zu mehreren informellen
würde. Die informelle Gruppe (Clique innerhalb der Klasse, Freundeskreis)
beruht vor allem auf den gefühlsmäßigen Beziehungen der Mitglieder
untereinander. Besteht die informelle Gruppe aus nur zwei Mitgliedern, so nennt
man sie Dyade (Ehe, Zweierbeziehung, Freundespaar). In ihr werden die
wechselseitigen Erwartungen und Hoffnungen, die Abhängigkeit des einen von
Entscheidungen, von Gefühlen des anderen besonders deutlich. Wird die eigene
Gruppe von dem Mitgliedern selbst positiv wahrgenommen, so spricht man von
einer Wir-Gruppe (Ingroup), wogegen die anderen, die der eigenen Gruppe nicht
angehören, als Die-Gruppe oder Outgroup eher negativ bewertet werden.
Schließlich spricht man von einer Bezugsgruppe, wenn ein Individuum sich an den
Eigenarten und Erwartungen dieser Gruppe misst.

\subsubsection{Abgrenzungen}
\begin{itemize}
	\item[Menge:] (\emph{Def.:}) Anzahl von Menschen, die zufällig zu
		einem bestimmten Zeitpunkt zusammen sind (instabil), \zB in
		einer Bahnhofshalle, im Supermarkt, ...

		Eine Menge kann werden zur:
		\begin{itemize}
			\item[Masse:] (\emph{Def.:}) große Zahl von Menschen auf
				engem Raum, einheitlich auf ein Ziel ausgerichtet
				(gefühlsmäßig); Massenphänomen: \emph{Panik} (Binnenstruktur
				einer Gruppe fehlt)
			\item[psycholog. Gruppe:] siehe Abschnitt
				\ref{subsec:Gruppenmerkmale}
		\end{itemize}
	\item[Klasse:] (\emph{Def.:})
		\begin{itemize}
			\item Anzahl von Menschen, die durch ein \emph{bestimmtes Merkmal}
				gekennzeichnet sind, \zB
				\begin{itemize}
					\item Frauen
					\item Männer
					\item Mercedesfahrer
				\end{itemize}
			\item Zuweisung zu einer \emph{sozialen Schicht:}
				\begin{itemize}
					\item Einkommen
					\item Wohngegend
					\item Herkunft, Hautfarbe
					\item Schulbildung
					\item Berufsausbildung
				\end{itemize}
		\end{itemize}
		\item[Verband:]
			\begin{itemize}
				\item Gewerkschaften
				\item Fußballverein
			\end{itemize}
			Charakteristika:
				\begin{itemize}
					\item gemeinsames Ziel
					\item offizieller Charakter
				\end{itemize}
\end{itemize}

\subsection{Entstehung von Gruppen}

\subsubsection{Die Ausgangssituation}
Gruppenfindung aufgrund eines Themas \(\longrightarrow\) erste Begegnung

\subsubsection{Die Phasen der Gruppenbildung}
Was wir oben beschrieben haben und leicht fortsetzen könnten, ist der Beginn
der Gruppenbildung. Die durch äußere Gegebenheiten an einem Ort zu einer
bestimmten Zeit versammelte Menge ist im Begriff, sich zur
sozialpsychologischen Gruppe zu formieren. Eines der üblichen Modelle
(\textsc{Tuckmann} 1965) beschreibt diesen Prozess in vier Phasen: Formierung,
Konflikt, Normierung und Arbeit.

In der im obigen Beispiel angedeuteten Formierungsphase wird die
Gesamtsituation von den einzelnen geprüft und nach dem jeweils angemessenen
eigenen Verhalten gesucht. Die Aufgaben und Ziele werden in etwa abgesteckt,
die eingebrachten Standpunkte aufgelockert, gemeinsame Methoden und Regeln ein
erstes Mal ausprobiert. Im Hinterkopf aber lauert die Angst vor der
Abhängigkeit von anderen oder vor einem Führer, und der Gedanke an eine
mögliche Flucht ist wach.

Auf die Formierungsphase folgt die Konfliktphase, in der sachliche
Meinungsverschiedenheiten innerhalb der Gruppe ausgetragen werden, häufig
allerdings auf einem recht persönliche und emotionalen Hintergrund. Man wehrt
sich gegen bestimmte Anforderungen, versucht die Grenzen des Möglichen
auszutesten, rebelliert gegen angemaßte Führerschaft. Die oft illusionäre
Harmonie der Formierungsphase weicht dem Konflikt um die gegenseitige Kontrolle
und um das richtige Maß an Intimität.

Übersteht die Gruppe diese Zerreißprobe, so tritt sie eher nüchtern in die
Normierungsphase ein, wo es gilt, eine von allen getragene Lösung der
aufgetretenen Probleme zu finden und diese zu verfestigen. Der
Gruppenzusammenhalt wird zusehends größer, das Gruppengefühl sicherer. Man kann
jetzt offen miteinander über Meinungen und Gefühle sprechen, die der anderen
verstehen und akzeptieren, sich gegenseitig im Sinne des Gruppenziels helfen
und sich unterstützen. Gruppennormen, von allen akzeptiert und beachtet,
erleichtern den Umgang miteinander; echte Kommunikation und Kooperation
zwischen den Gruppenmitgliedern wird möglich.

Die Arbeitsphase als letztes Glied bei der Entstehung von Gruppen kann
beginnen. Alle Mitglieder sind bereit zu produktiver und effektiver Arbeit. Die
Rollen sind funktional verteilt; man strengt sich an, in flexibler Kooperation
das gemeinsame Problem zu lösen, ungehindert von zwischenmenschlichen Querelen.
Die Beziehungen untereinander stehen ganz im Dienste der übernommenen Aufgabe.

Diese Phasen der Gruppenbildung kann man bei vielen Gelegenheiten mehr oder
weniger deutlich beobachten, so, wenn eine Schulklasse neu zusammengewürfelt
wird, etwa beim Übergang ins Gymnasium, oder wenn Jungen und Mädchen aus der
Nachbarschaft sich zu einer Clique zusammenschließen, wenn im Sportverein eine
neue Abteilung gegründet wird oder wenn durch betriebliche Neuorganisation
bisherige Zugehörigkeiten auseinandergerissen werden, auch wenn Studenten zur
Vorbereitung ihres Staatsexamens Arbeitsgruppen bilden oder wenn eine Gruppe
von Schülern nach dem Abitur für mehrere Wochen auf gemeinsame große Fahrt geht.

\subsubsection{Bedingungen des Gruppengeschehens}
Das Gruppengeschehen in den einzelnen Phasen wird im wesentlichen von drei
wichtigen Bedingungen bestimmt: von den Aktivitäten auf ein bestimmtes Ziel
hin, von den sachgeleiteten Kontakten zwischen den Mitgliedern und von der
Sympathie, die sie füreinander empfinden.

In der Gruppe wird ein bestimmtes Verhalten von einem einzelnen eingebracht,
von den anderen wahrgenommen und bewertet. Entspricht es den Zielvorstellungen
der anderen, so löst es bei ihnen auch Aktivitäten aus. Diese beziehen sich auf
sachliche Gegebenheiten der Umwelt und schließen gleichzeitig alle Beteiligten
als Personen mit ein. Die Klärung der Beziehungen zwischen den Mitgliedern wird
damit zum ersten und wichtigsten Ziel der Gruppe.

Die damit etablierten Kontakte stellen ein wichtiges Mittel zur sachlichen
Zielerreichung dar. Man anerkennt die Sachkompetenz des einen oder die
sozioemotionale Kompetenz des anderen, man ordnet sich freiwillig dem mit den
bewährten Führungsqualitäten unter und löst so in optimaler Arbeitsteilung die
gestellten Aufgaben.

Nicht zuletzt hängt das, was jeder einzelne dazu beitragen kann, von der
Sympathie ab, die er in seiner Gruppe spürt. Man nennt diese Zusammenhänge die
affektive Binnenstruktur einer Gruppe. Sie prägt den Zusammenhalt, vermittelt
den Mitgliedern ein Wir-Gefühl, ist aber auch verantwortlich für
Polarisierungen, die Außenseiter oder gar Feinde schaffen.

Die drei Bedingungen von Aktivität, Kontakt und Sympathie sind, wie leicht
einzusehen ist, nicht voneinander unabhängig; sie üben eine starke
Wechselwirkung aufeinander aus. So besagt die \textsc{Homans}sche Regel, dass
verstärkter Kontakt zu größerer Sympathie und umgekehrt größere Sympathie zu
häufigerem Kontakt führt. Leuchtet dieser Zusammenhang auf den ersten Blick
ohne weiteres ein, so haben doch entsprechende Untersuchungen nur eine
partielle Bestätigung erbracht. Erhöhter Kontakt führt zu einer schnelleren
Klärung von Sympathie-Antipathie-Beziehungen, selbstverständlich abhängig von
den jeweiligen Rahmenbedingungen. Immerhin: Hatte man früher zwei beinahe
Unbekannte miteinander verheiratet, so pflegte man zu sagen, dass im Laufe der
Zeit die Liebe sich von allein einstellen würde, während man heute eher dazu
neigt, in einer Probe-Ehe die Sympathie-Antipathie-Beziehungen rechtzeitig
auszuloten. Auch zu den Wechselwirkungen Sympathie-Aktivität (wie bei der
„Aktion Sühnezeichen“, in der seit dem Ende des letzten Weltkrieges deutsche
Jugendliche mit Jugendlichen anderer, im Krieg von den Deutschen besetzter
Länder über Wochen und Monate zusammenarbeiten, um \zB einen
Soldatenfriedhof neu zu gestalten und anschließend zu pflegen) und zu den
Wechselwirkungen Aktivität-Kontakt gibt es zahlreiche interessante
Untersuchungen, die -- wie zu erwarten -- allerdings nie einfache
Wenn-Dann-Bezüge bestätigen.

\end{document}
