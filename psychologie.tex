% !TeX encoding = UTF-8
% !TeX spellcheck = de_DE
\documentclass[12pt]{scrartcl}

\usepackage[utf8]{inputenc}
\usepackage[T1]{fontenc}
\usepackage[ngerman]{babel}
\usepackage{ulem}

\begin{document}
\tableofcontents \pagebreak

\section{Gruppe}

\subsection{Einführung und Überblick}
Die \emph{Gruppenpsychologie} (Teil der \emph{Sozialpsychologie}) beschäftigt
sich mit den \emph{Auswirkungen} der \emph{sozialen Umwelt} auf den
\emph{einzelnen Menschen}.

\paragraph{zwei Teildisziplinen:}
\begin{enumerate}
	\item Gruppenforschung (empirisch/experimentell arbeitend)
	\item angewandte Gruppendynamik
\end{enumerate}

\paragraph{Ziel:} überdauernde Verhaltensänderungen

\subsubsection*{zu 1:} seit Mitte des letzten Jahrhunderts (Lewin: 1890-1947),
Höhepunkt um und nach dem 2. Weltkrieg (zwischen 1950 und 1975 wichtigster
Zweig der Sozialpsychologie)

\paragraph{Themen:}
\begin{itemize}
	\item Gruppendynamik
	\item Gruppenbildung/Motive
	\item Gruppenzusammenhalt
	\item Gruppenspezifische Verhaltensweisen (Zwänge?)
	\item Rangordnung
	\item Kommunikation
	\item Vor- und Nachteile
	\item Kooperation
	\item Entstehen von Einstellungen/Urteilsvermögen/Wahrnehmung
\end{itemize}

\paragraph{Ziel:} Verhaltensänderungen (Vermeidung politischer Katastrophen: J.
L. Moreno, 1966)

\subsubsection*{zu 2:} \emph{Gruppendynamische Experimente}

\end{document}
