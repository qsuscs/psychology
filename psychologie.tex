% !TeX encoding = UTF-8
% !TeX spellcheck = de_DE
% !TEX program = xelatex
% vim:ts=4
\documentclass[12pt]{scrartcl}

\usepackage{ifxetex}
\ifxetex
	\usepackage{polyglossia}
	\setmainlanguage[spelling = new, babelshorthands = true]{german}
	\usepackage{fontspec}
\else
	\usepackage[ngerman]{babel}
	\usepackage[utf8]{inputenc}
	\usepackage[T1]{fontenc}
\fi
\usepackage{hyperref}

\usepackage{xspace}
\DeclareRobustCommand{\zB}{z.\,B.\xspace}

\title{Psychologie Kursstufe II}
\author{B.~Hanke-Hassel\\Mit- und Abschrift von Thomas~Schneider}
\date{} % Datum gibt hier keinen Sinn, daher leer

\begin{document}
\maketitle
\tableofcontents \pagebreak

\section{Gruppe}

\subsection{Einführung und Überblick}
Die \emph{Gruppenpsychologie} (Teil der \emph{Sozialpsychologie}) beschäftigt
sich mit den \emph{Auswirkungen} der \emph{sozialen Umwelt} auf den
\emph{einzelnen Menschen}.

\paragraph{zwei Teildisziplinen:}
\begin{enumerate}
	\item Gruppenforschung (empirisch/experimentell arbeitend)
	\item angewandte Gruppendynamik
\end{enumerate}

\paragraph{Ziel:} überdauernde Verhaltensänderungen

\subsubsection*{zu 1:} seit Mitte des letzten Jahrhunderts (Lewin: 1890-1947),
Höhepunkt um und nach dem 2. Weltkrieg (zwischen 1950 und 1975 wichtigster
Zweig der Sozialpsychologie)

\paragraph{Themen:}
\begin{itemize}
	\item Gruppendynamik
	\item Gruppenbildung/Motive
	\item Gruppenzusammenhalt
	\item Gruppenspezifische Verhaltensweisen (Zwänge?)
	\item Rangordnung
	\item Kommunikation
	\item Vor- und Nachteile
	\item Kooperation
	\item Entstehen von Einstellungen/Urteilsvermögen/Wahrnehmung
\end{itemize}

\paragraph{Ziel:} Verhaltensänderungen (Vermeidung politischer Katastrophen:
J.~L.~Moreno, 1966)

\subsubsection*{zu 2:}
\emph{Gruppendynamische Experimente:} Um Erkenntnisse, Einstellungen zu
erreichen, d.\,h. Verhaltensänderungen der Gruppenmitglieder. Solche
Experimente sind planmäßig aufgebaute und durchgeführte Gruppenprozesse.

\subsection{Definition und Arten}

\subsubsection{Gruppenmerkmale} \label{subsec:Gruppenmerkmale}
Für den sozialpsychologischen Begriff der Gruppe sind die wechselseitigen
Erwartungen der Mitglieder untereinander konstituierend: Als Mitglied der
Gruppe glaube ich sicher zu sein, wie die anderen Mitglieder mich sehen; ich
selbst habe mir eine genaue Vorstellung aller anderen erworben; gleichzeitig
weiß ich aber auch, dass alle anderen über diese Dinge Bescheid wissen oder
wenigstens feste Vermutungen darüber haben. So wird das Verhalten der
Individuen in der Gruppe von solchen Hypothesen auf verschiedenen Ebenen
gesteuert. Neben dem genannten Hauptmerkmal der Interdependenz, durch das
gleichzeitig die Zahl der Mitglieder auf 8 bis 12 beschränkt ist, gibt es eine
Reihe weiterer Merkmale, die für eine sozialpsychologische Gruppe von Bedeutung
sind:
\begin{itemize}
	\item Interaktionen und gefühlsmäßige Wechselbeziehungen aller Mitglieder
		untereinander
	\item Strukturierung durch allgemein anerkannte Positionen und Rollen zur
		Organisation von Abläufen innerhalb der Gruppe
	\item Gruppenziel und gemeinsame Normen auf gruppenrelevantem Gebiet
	\item Zusammengehörigkeitsgefühl im Sinne von Gruppenbewusstsein als
		Identifikation der Mitglieder mit der Gruppe und als einheitliches
		Verhalten gegenüber der Außenwelt
	\item Relative Dauer der Gruppenzugehörigkeit und Erleben von Kontinuität
		durch Traditionen und Gewohnheiten.
\end{itemize}
Kein brauchbares Merkmal der sozialpsychologischen Gruppe ist die sogenannte
„Gruppenseele“, auch wenn man häufig feststellen kann, dass Gruppen ihren
Charakter über lange Zeit und trotz Mitgliederwechsel bewahren. Damit würde der
Gruppe eine Eigenschaft zugeschrieben, die sie nicht haben kann, weil sie als
Person nicht existiert, sondern lediglich als Summe ihrer Mitglieder. Diese
bewirken durch ihr aufeinander bezogenes Verhalten in Vorstellungen,
Hoffnungen, Konflikten einen dynamischen, affektiv und kognitiv gesteuerten
Prozess, der vom Außenstehenden ganzheitlich wahrgenommen und als typisch für
diese Gruppe gesehen wird (\zB Mitglieder einer Jugendbande, Schüler einer
bestimmten Klasse).

\subsubsection{Gruppeneinteilung}
Der Versuch, die vielen sozialpsychologischen Gruppen systematisch einzuteilen,
ist schwierig, weil sich nur selten eindeutige Zuordnungen treffen lassen. In
der Regel unterscheidet man zunächst die Primärgruppen (Familie, Großfamilie,
Nachbarschaft), zu denen jeder seit seiner Kindheit gehört und denen er
lebenslang verhaftet bleibt, von Sekundärgruppen, denen man sich erst später
und häufig auch nur vorübergehend anschließt (Schulkameraden, Kollegen).
Sekundärgruppen werden ihrerseits in formelle und informelle Gruppen
aufgeteilt. Formelle Gruppen entstehen durch äußere Bedingungen: Eine
Schulklasse wird neu gebildet, im Studium arbeiten verschiedene Studenten im
gleichen Labor, eine Betriebsabteilung wird geschaffen. Es gibt keine formelle
Gruppe, die nicht im Lauf der Zeit zur informellen oder zu mehreren informellen
würde. Die informelle Gruppe (Clique innerhalb der Klasse, Freundeskreis)
beruht vor allem auf den gefühlsmäßigen Beziehungen der Mitglieder
untereinander. Besteht die informelle Gruppe aus nur zwei Mitgliedern, so nennt
man sie Dyade (Ehe, Zweierbeziehung, Freundespaar). In ihr werden die
wechselseitigen Erwartungen und Hoffnungen, die Abhängigkeit des einen von
Entscheidungen, von Gefühlen des anderen besonders deutlich. Wird die eigene
Gruppe von dem Mitgliedern selbst positiv wahrgenommen, so spricht man von
einer Wir-Gruppe (Ingroup), wogegen die anderen, die der eigenen Gruppe nicht
angehören, als Die-Gruppe oder Outgroup eher negativ bewertet werden.
Schließlich spricht man von einer Bezugsgruppe, wenn ein Individuum sich an den
Eigenarten und Erwartungen dieser Gruppe misst.

\subsubsection{Abgrenzungen}
\begin{itemize}
	\item[Menge:] (\emph{Def.:}) Anzahl von Menschen, die zufällig zu
		einem bestimmten Zeitpunkt zusammen sind (instabil), \zB in
		einer Bahnhofshalle, im Supermarkt, ...

		Eine Menge kann werden zur:
		\begin{itemize}
			\item[Masse:] (\emph{Def.:}) große Zahl von Menschen auf
				engem Raum, einheitlich auf ein Ziel ausgerichtet
				(gefühlsmäßig); Massenphänomen: \emph{Panik} (Binnenstruktur
				einer Gruppe fehlt)
			\item[psycholog. Gruppe:] siehe Abschnitt
				\ref{subsec:Gruppenmerkmale}
		\end{itemize}
	\item[Klasse:] (\emph{Def.:})
		\begin{itemize}
			\item Anzahl von Menschen, die durch ein \emph{bestimmtes Merkmal}
				gekennzeichnet sind, \zB
				\begin{itemize}
					\item Frauen
					\item Männer
					\item Mercedesfahrer
				\end{itemize}
			\item Zuweisung zu einer \emph{sozialen Schicht:}
				\begin{itemize}
					\item Einkommen
					\item Wohngegend
					\item Herkunft, Hautfarbe
					\item Schulbildung
					\item Berufsausbildung
				\end{itemize}
		\end{itemize}
		\item[Verband:]
			\begin{itemize}
				\item Gewerkschaften
				\item Fußballverein
			\end{itemize}
			Charakteristika:
				\begin{itemize}
					\item gemeinsames Ziel
					\item offizieller Charakter
				\end{itemize}
\end{itemize}

\subsection{Entstehung von Gruppen}

\subsubsection{Die Ausgangssituation}
Gruppenfindung aufgrund eines Themas \(\longrightarrow\) erste Begegnung

\subsubsection{Die Phasen der Gruppenbildung}
Was wir oben beschrieben haben und leicht fortsetzen könnten, ist der Beginn
der Gruppenbildung. Die durch äußere Gegebenheiten an einem Ort zu einer
bestimmten Zeit versammelte Menge ist im Begriff, sich zur
sozialpsychologischen Gruppe zu formieren. Eines der üblichen Modelle
(\textsc{Tuckmann} 1965) beschreibt diesen Prozess in vier Phasen: Formierung,
Konflikt, Normierung und Arbeit.

In der im obigen Beispiel angedeuteten Formierungsphase wird die
Gesamtsituation von den einzelnen geprüft und nach dem jeweils angemessenen
eigenen Verhalten gesucht. Die Aufgaben und Ziele werden in etwa abgesteckt,
die eingebrachten Standpunkte aufgelockert, gemeinsame Methoden und Regeln ein
erstes Mal ausprobiert. Im Hinterkopf aber lauert die Angst vor der
Abhängigkeit von anderen oder vor einem Führer, und der Gedanke an eine
mögliche Flucht ist wach.

Auf die Formierungsphase folgt die Konfliktphase, in der sachliche
Meinungsverschiedenheiten innerhalb der Gruppe ausgetragen werden, häufig
allerdings auf einem recht persönliche und emotionalen Hintergrund. Man wehrt
sich gegen bestimmte Anforderungen, versucht die Grenzen des Möglichen
auszutesten, rebelliert gegen angemaßte Führerschaft. Die oft illusionäre
Harmonie der Formierungsphase weicht dem Konflikt um die gegenseitige Kontrolle
und um das richtige Maß an Intimität.

Übersteht die Gruppe diese Zerreißprobe, so tritt sie eher nüchtern in die
Normierungsphase ein, wo es gilt, eine von allen getragene Lösung der
aufgetretenen Probleme zu finden und diese zu verfestigen. Der
Gruppenzusammenhalt wird zusehends größer, das Gruppengefühl sicherer. Man kann
jetzt offen miteinander über Meinungen und Gefühle sprechen, die der anderen
verstehen und akzeptieren, sich gegenseitig im Sinne des Gruppenziels helfen
und sich unterstützen. Gruppennormen, von allen akzeptiert und beachtet,
erleichtern den Umgang miteinander; echte Kommunikation und Kooperation
zwischen den Gruppenmitgliedern wird möglich.

Die Arbeitsphase als letztes Glied bei der Entstehung von Gruppen kann
beginnen. Alle Mitglieder sind bereit zu produktiver und effektiver Arbeit. Die
Rollen sind funktional verteilt; man strengt sich an, in flexibler Kooperation
das gemeinsame Problem zu lösen, ungehindert von zwischenmenschlichen Querelen.
Die Beziehungen untereinander stehen ganz im Dienste der übernommenen Aufgabe.

Diese Phasen der Gruppenbildung kann man bei vielen Gelegenheiten mehr oder
weniger deutlich beobachten, so, wenn eine Schulklasse neu zusammengewürfelt
wird, etwa beim Übergang ins Gymnasium, oder wenn Jungen und Mädchen aus der
Nachbarschaft sich zu einer Clique zusammenschließen, wenn im Sportverein eine
neue Abteilung gegründet wird oder wenn durch betriebliche Neuorganisation
bisherige Zugehörigkeiten auseinandergerissen werden, auch wenn Studenten zur
Vorbereitung ihres Staatsexamens Arbeitsgruppen bilden oder wenn eine Gruppe
von Schülern nach dem Abitur für mehrere Wochen auf gemeinsame große Fahrt geht.

\subsubsection{Bedingungen des Gruppengeschehens}
Das Gruppengeschehen in den einzelnen Phasen wird im wesentlichen von drei
wichtigen Bedingungen bestimmt: von den Aktivitäten auf ein bestimmtes Ziel
hin, von den sachgeleiteten Kontakten zwischen den Mitgliedern und von der
Sympathie, die sie füreinander empfinden.

In der Gruppe wird ein bestimmtes Verhalten von einem einzelnen eingebracht,
von den anderen wahrgenommen und bewertet. Entspricht es den Zielvorstellungen
der anderen, so löst es bei ihnen auch Aktivitäten aus. Diese beziehen sich auf
sachliche Gegebenheiten der Umwelt und schließen gleichzeitig alle Beteiligten
als Personen mit ein. Die Klärung der Beziehungen zwischen den Mitgliedern wird
damit zum ersten und wichtigsten Ziel der Gruppe.

Die damit etablierten Kontakte stellen ein wichtiges Mittel zur sachlichen
Zielerreichung dar. Man anerkennt die Sachkompetenz des einen oder die
sozioemotionale Kompetenz des anderen, man ordnet sich freiwillig dem mit den
bewährten Führungsqualitäten unter und löst so in optimaler Arbeitsteilung die
gestellten Aufgaben.

Nicht zuletzt hängt das, was jeder einzelne dazu beitragen kann, von der
Sympathie ab, die er in seiner Gruppe spürt. Man nennt diese Zusammenhänge die
affektive Binnenstruktur einer Gruppe. Sie prägt den Zusammenhalt, vermittelt
den Mitgliedern ein Wir-Gefühl, ist aber auch verantwortlich für
Polarisierungen, die Außenseiter oder gar Feinde schaffen.

Die drei Bedingungen von Aktivität, Kontakt und Sympathie sind, wie leicht
einzusehen ist, nicht voneinander unabhängig; sie üben eine starke
Wechselwirkung aufeinander aus. So besagt die \textsc{Homans}sche Regel, dass
verstärkter Kontakt zu größerer Sympathie und umgekehrt größere Sympathie zu
häufigerem Kontakt führt. Leuchtet dieser Zusammenhang auf den ersten Blick
ohne weiteres ein, so haben doch entsprechende Untersuchungen nur eine
partielle Bestätigung erbracht. Erhöhter Kontakt führt zu einer schnelleren
Klärung von Sympathie-Antipathie-Beziehungen, selbstverständlich abhängig von
den jeweiligen Rahmenbedingungen. Immerhin: Hatte man früher zwei beinahe
Unbekannte miteinander verheiratet, so pflegte man zu sagen, dass im Laufe der
Zeit die Liebe sich von allein einstellen würde, während man heute eher dazu
neigt, in einer Probe-Ehe die Sympathie-Antipathie-Beziehungen rechtzeitig
auszuloten. Auch zu den Wechselwirkungen Sympathie-Aktivität (wie bei der
„Aktion Sühnezeichen“, in der seit dem Ende des letzten Weltkrieges deutsche
Jugendliche mit Jugendlichen anderer, im Krieg von den Deutschen besetzter
Länder über Wochen und Monate zusammenarbeiten, um \zB einen
Soldatenfriedhof neu zu gestalten und anschließend zu pflegen) und zu den
Wechselwirkungen Aktivität-Kontakt gibt es zahlreiche interessante
Untersuchungen, die -- wie zu erwarten -- allerdings nie einfache
Wenn-Dann-Bezüge bestätigen.

\subsection{Gruppe und Rolle}

\subsubsection{Die soziale Rolle}
Hat sich eine Gruppe einmal konstituiert, so nimmt jedes Mitglied innerhalb der
Gruppe eine bestimmte soziale Position ein, zu der mehr oder weniger
festgelegte Funktionen im Sinne von Aufgaben und Pflichten, aber auch von
Rechten oder gar von Vorrechten gehören. Solche spezifischen Verhaltensweisen
werden sowohl von dem Inhaber der Position als auch von den übrigen
Gruppenmitgliedern voll gebilligt und anerkannt. Damit ist die soziale Rolle
eines Individuums in der Gruppe beschrieben als die Gesamtheit von
Verhaltensweisen, die mit der Position in der Gruppe verknüpft sind.
\textsc{Hartley} (1969) definiert ausführlicher die soziale Rolle als die
strukturierte Gesamtheit aller Erwartungen, die sich auf die Aufgaben, das
Benehmen, die Gesinnungen, die Werte und die wechselseitigen Beziehungen einer
Person richten, die eine spezifische Gruppenposition innehat und in der Gruppe
eine bestimmte Funktion erfüllen muss. Stärker als in unserer vorläufigen
Beschreibung wird in dieser Definition der sozialen Rolle deutlich, dass sie
mehr durch das geprägt wird, was die anderen Gruppenmitglieder von ihrem Träger
erwarten, als durch dessen tatsächliches Verhalten, das nur zeigt, in welchem
Maße er seine Rolle ausfüllt. Verdeutlichen wir uns diesen Zusammenhang am
Beispiel des sogenannten Klassenkaspers, also eines Schülers, der zur Freude
seiner Kameraden den Unterricht immer wieder stärt und alle, oft auch den
Lehrer, zum Lachen bringt. Die Mitglieder erwarten vom Klassenkasper
irgendwelche Streiche. Jedes Mal, wenn er etwas anstellt, erfährt er Zuwendung
und wird so in seinem Verhalten verstärkt. Achtung und Anerkennung sind ihm
sicher, wohingegen seine Bemühungen, im Unterricht ernsthaft mitzuarbeiten, gar
nicht registriert werden. Schließlich akzeptiert er die von ihm erwartete Rolle
und versuche, sie so gut es geht auszufüllen.

\subsubsection{Rangordnung}
In jeder sozialpsychologischen Gruppe ergeben Prestige und Ansehen der
einzelnen Mitglieder eine Rangordnung, durch die die Individuen einander
hierarchisch über- oder untergeordnet sind. Diese von allen Beteiligten
anerkannte und beachtete sogenannte Hackordnung ist auf dem Hühnerhof angeblich
besonders gut zu beobachten. Wenn dort das Futter knapp ist, frisst sich
zunächst der Hahn satt und verhindert durch Hacken mit dem Schnabel, dass
irgendein anderes Federvieh sich an seinem Mahl beteiligt. Ist er satt, so
beginnt die ranghöchste Henne zu fressen; bezogen auf den übrigen Hühnerhof,
verhält sie sich genau so wie vorher der Hahn ihr gegenüber. Gibt es zu wenig
Futter, bekommen die rangniedrigsten Tiere, meist Junghähnchen oder
altersschwache Hühner, nichts mehr zu fressen. Anders als auf dem Hühnerhof,
der ja eingezäunt ist, könnte bei sozialpsychologischen Gruppen eine solche
einfache Hierarchie bewirken, dass die jeweils benachteiligten Mitglieder aus
der Gruppe wegblieben, was zu dauernden Veränderungen im Gruppengefüge führen
würde.

Genau das Gegenteil ist aber zu beobachten: Sozialpsychologische Gruppen sind
in sich sehr stabil und haben eine ausgeprägte und relativ feste
Gruppenstruktur. Dies kommt dadurch zustande, dass sich in der Gruppe zwei,
meistens mehr Rangordnungen überlagern und somit jedes Mitglied mehr als eine
gruppenspezifische Rolle übernimmt. Wenn sich eine Anzahl von Jungen zu einer
Gruppe zusammenschließt, um zusammen zu kicken, so überlagert sich dort der
Tüchtigkeitsrang und der Beliebtheitsrang eines einzelnen zu einer festen
Gruppenstruktur. Während der Tüchtigkeitsrang durch Funktionen (\zB
Stürmer), Ziele (\zB Tore schießen) und Normen (\zB Fairness) bestimmt
wird, ist der Beliebheitsrang abhängig von der integrativen Kraft des einzelnen
Mitglieds, die den Zusammenhalt fördert, und von seiner Fähigkeit, durch ein
entsprechendes Wort eine affektiv schwierige Situation zu bereinigen
(Atmosphäre). Der Tüchtigste wird selten auch gleichzeitig der Beliebteste
sein, der am wenigsten Tüchtige nie auch der Unbeliebteste. So werden allen
Mitgliedern hinsichtlich ihrer Rolle in der Gruppe ganz bestimmte Erwartungen,
sowohl bezogen auf Tüchtigkeit als auch bezogen auf Beliebtheit,
entgegengebracht, und sie müssen versuchen, diese zu erfüllen. Labil kann die
sogenannte Gruppenstruktur durch das Ausscheiden eines bisherigen oder das
Hinzukommen eines neuen Mitgliedes werden, auch durch Veränderungen in der
Zielsetzung der Gruppe. Dann wird es nämlich notwendig, die Rollen mehr oder
weniger neu zu definieren, und jeder einzelne muss prüfen, ob er in der Lage
ist, sie so zu akzeptieren.

\subsubsection{Rolle und sozialer Status}
Jeder Mensch durchläuft in verschiedenen Gruppen während seines Lebens eine
Reihe von Rollen, die mit seiner Stellung in diesen Gruppen zusammenhängen.
Selbst in einfachen Kulturen gibt es dafür verschiedene Möglichkeiten: Alter
und Geschlecht (\zB Kind, Mann, Frau), Familie, Beruf, persönliches Prestige
(Bildung, Geld) und Mitgliedschaft in Interessengruppen (\zB Gewerkschaften,
Religionsgemeinschaften). Ein Teil dieser Rollen ist eher naturbedinge (\zB
Kindrolle, Geschlechtsrolle) oder gesellschaftlich genau definiert, d.\,h.
durch eindeutige Verhaltenserwartungen festgelegt (\zB Mutter, Arzt, Richter),
ein anderer Teil ist eher unklar, das Verhalten des Rollenträgers ist nicht
genau beschrieben (\zB Hauseigentümer, Privatgelehrter, Jugendliche).

Je offener eine Rolle sich darstellt, um so mehr hängt sie vom sozialen Status
des einzelnen innerhalb seiner Gruppe ab. Die Determinanten für den Status sind
alle schon genannten Rollen als Schüler, Führer, Hausfrau, Vater, Lehrer oder
Arzt und viele andere mehr. Diese Aufzählung macht aber auch deutlich, dass die
Handlungsspielräume ihrerseits wieder nicht beliebig sind, sondern dass sie an
sozial definierte und vermittelte Grenzen stoßen: Ein Arzt ist zur
Hilfeleistung verpflichtet, eine Lehrer muss den Schülern gegenüber gerecht
sein, eine Mutter darf ihr Kind nicht prügeln. Rollen werden einem Individuum
entweder zugeschrieben (Mann, Frau, Kind), oder sie werden durch bestimmte
Qualifikationen von diesem erworben (Richter, Arzt).

Personen, die den Erwartungen an ihre Rolle einerseits entsprechen wollen,
andererseits aber Schwierigkeiten haben, sich mit ihr zu identifizieren, wählen
als Ausweg häufig eine Form der Rollendistanzierung. Man versucht also, den
Handlungserwartungen zu entsprechen, ironisiert aber \zB gleichzeitig das
eigene Tun. Man kann auch das eigene Handeln übertreiben und damit die Rolle
überbetonen oder andeuten, dass alles nicht so ernst gemeint sei. Jemand geht
dadurch in die innere Emigration, dass er zwar äußerlich die
Handlungserwartungen erfüllt, innerlich sich aber von allem weitgehend
distanziert. Nicht zuletzt kann die Rollendistanzierung bis ins Krankhafte
gesteigert werden, indem ein Individuum sich aus seiner Rolle in eine Welt des
Traumes oder der Fantasie flüchtet.

\subsubsection{Rollenkonflikte}
Die Rollendistanzierung könnte ein erster Hinweis auf einen sich anbahnenden
Rollenkonflikt sein. Rollenkonflikte entstehen, wenn Rollen unklar definiert
sind oder wenn es dem Individuum nicht gelinkt, sich mit den an die Rolle
geknüpften Verhaltenserwartungen zu identifizieren. Man unterscheidet
Intrarollenkonflikte als widersprüchliche Erwartungen innerhalb einer Rolle und
Interrollenkonflikte, die sich zwischen verschiedenen Rollen ein- und derselben
Person auftun. Ein \emph{Intrarollenkonflikt} liegt \zB vor, wenn ein Lehrer
einerseits sich verpflichtet fühlt, nach Lehrplan zu unterrichten, er
andererseits aber überzeugt ist, dass es für die Persönlichkeitsbildung seiner
Schüler viel wichtiger wäre, über ein aktuelles -- im Lehrplan nicht
vorgesehenes -- Thema ausführlich zu sprechen. Der derzeit wohl am häufigsten
beschriebene \emph{Interrollenkonflikt} ist der der Frau, die ihre Rollen als
emanzipierte Frau, als Ehefrau, als Mutter, als Hausfrau und als berufstätige
Frau nicht unter einen Hut bringen kann.

Die Lösung von Rollenkonflikten wird durch eine größere Freiheit der
Interpretation sowohl erleichtert als erschwert. Erleichtert insofern, als
innerhalb eines weiten Spektrums, das von der Gruppe noch akzeptiert wird,
Verhaltensmöglichkeiten gefunden werden können, die sowohl die Erwartungen der
anderen abdecken als auch die eigenen Bedürfnisse befriedigen. Die Rolle als
Hausfrau im obigen Beispiel kann dadurch, dass die übrigen im Haushalt lebenden
Personen Teilaufgaben übernehmen, soweit zurückgenommen werden, dass genügend
Raum für die angestrebten und erwünschten beruflichen Verpflichtungen bleibt.
Erschwert wird die Situation durch den Zwang zur persönlich verantworteten
Interpretation der eigenen Rolle. Bleiben wir beim genannten Beispiel: Solange
es für ein Mädchen klar war, dass es eines Tages die Rollen der Hausfrau und
der Mutter, beide verbunden mit sehr genauen Verhaltenserwartungen, die auch
von niemand bezweifelt wurden, übernehmen würde, brauchte es sich um die
Ausgestaltung seines zukünftigen Lebens keine Gedanken zu machen,
Rollenkonflikte nicht zu befürchten.

Eine wichtige Aufgabe des Sozialisationsprozesses in den verschiedenen
Gruppierungen besteht demnach darin, junge Menschen mit den Qualifikationen
auszustatten, die notwendig sind, um innerhalb der angemuteten Rollen
erfolgreich zu handeln, um Rollen entsprechend den eigenen Überzeugungen zu
interpretieren und um Rollenkonflikte ertragen und auch lösen zu können.

\subsection{Werte und Normen}

\subsubsection{Arten}
In unserer Gesellschaft insgesamt und in jeder Gruppierung in ihr gibt es
bestimmte Verhaltensregeln, die das Zusammenleben bestimmen. Sie sind einander
über- und untergeordnet und werden in formellen Großgruppen Gesetze, Sitten
oder Bräuche, in informellen Kleingruppen Normen oder Regeln genannt.

Normen einer informellen Gruppe, meist ungeschrieben und nicht besonders
eindeutig formuliert, dienen der Erleichterung der Zusammenarbeit in der Gruppe
und damit dem Gruppenziel; sie fördern den Zusammenhalt durch die Reduzierung
von Konflikten und lassen die Mitglieder, deren Verhalten für jeden einzelnen
durchsichtig und prognostizierbar wird, zu einer Einheit zusammenwachsen. Es
gibt Normen, die nur für bestimmte Positionen innerhalb der Gruppe gelten (\zB
für den Klassenkasper), und generelle Normen, die für alle Mitglieder
verbindlich sind (\zB: „Du darfst nicht petzen“) und das Maß an Solidarität
bestimmen, das eine Gruppe für notwendig erachtet. Die Normen geben in der
Regel nicht ein bestimmtes erwünschtes Verhalten vor, sondern sie beschreiben
eher einen Verhaltensbereich, der von der Gruppe noch akzeptiert wird.

Die Toleranzgrenze ist für Handlungen auf gruppenrelevantem Gebiet eher eng
gezogen, insbesondere dann, wenn sich die Gruppe bedroht fühlt; sie ist eher
weit, wenn dem Verhalten eine geringe Bedeutung für die Gruppe zugemessen wird.
Sie hängt aber auch vom Status des einzelnen Mitglieds ab; je geringer dieser
ist, um so weniger kann es sich bestimmte Normverletzungen leisten.

\subsubsection{Internalisierung}
Im Laufe der Zeit erwachsen aus Gruppennormen eigene Anschauungen und Meinungen
der Mitglieder; man spricht von Internalisierung der Normen durch das
Individuum. Die so erworbenen und zum Teil der eigenen Persönlichkeit
gewordenen Wertekonzepte werden sowohl innerhalb als auch außerhalb der Gruppe
angewendet.

Bei den Untersuchungen von Mayo (1945) in den Hawthorne Western Electric Werken
wurden für eine Arbeitsgruppe die Arbeitsbedingungen verschiedentlich verändert
und die Veränderungen zuvor mit den Mitgliedern der Gruppe besprochen. So
führten die Verbesserung der Beleuchtungsverhältnisse, die Vergößerung der
Pausen, das Angebot von Mahlzeiten während der Pausen, die Verkürzung der
täglichen Arbeitszeit und Veränderungen im Arbeitsablauf und an den
Arbeitsplätzen jeweils zu Leistungssteigerungen, verglichen mit einer
Kontrollgruppe, die dieselben Aufgaben zu erledigen hatte, ohne dass die
Arbeitsbedingungen verändert wurden. Zum großen Erstaunen der begleitenden
Wissenschaftler blieb der erreichte hohe Leistungsstandard der Versuchsgruppe
auch dann noch erhalten, als am Ende der Versuchsphase rückgängig gemacht
wurden und die Arbeitsverhältnisse wieder genau denen der Kontrollgruppe
glichen.

Für unseren Zusammenhang interessiert vor allem, dass das Wertekonzept der
Versuchsgruppe hauptsächlich von Solidarität bestimmt war: Die Mitglieder
schützten sich vor überhöhtem Arbeitsdruck durch das Festlegen einer
angemessenen Arbeitsleistung, die von niemand wesentlich über- oder
unterschritten werden durfte. Sie legten von sich aus einen bestimmten
Qualitätsstandard für die geleistete Arbeit fest. Das Weitergeben von
Informationen nach außen, „Radfahren“ bei der Betriebsleitung und Angeberei vor
den anderen Mitgliedern der Arbeitsgruppe waren verpönt. Damit wird deutlich,
dass Wertekonzepte in Gruppen das Erreichen der Gruppenziele sichern helfen,
wenn vielleicht auch eine Nivellierung der Leistungen der einzelnen damit
verbunden ist. Sie fördern den Zusammenhalt der Gruppe durch Abgrenzungen nach
außen und durch den Abbau von Konfliktpotential nach innen.

Treffen für die einzelne Person zwei verschiedene Wertsysteme aufeinander, von
denen das eine sich aus Normen der formellen Gruppe (im Beispiel: offizielle
Vorschriften des Betriebs) rekrutiert, das andere der informellen Gruppe
entstammt, so erweisen sich die informellen Normen meist als stärker. So wird
ein Lehrer feststellen, dass ein Schüler, mit dem er ein Einzelgespräch über
dessen unmögliches Verhalten im Unterricht geführt hat, sich durchaus
einsichtig zeigt und Besserung gelobt (Regeln des formellen Systems). Schon in
der nächsten Stunde benimmt sich aber der Schüler unter dem Erwartungsdruck
seiner Kameraden wie immer (Regeln des informellen Systems), was dann der
Lehrer als besonders schwerwiegenden Vertrauensbruch empfindet.

\subsubsection{Soziale Kontrolle}
Damit wird aber schon etwas weiteres deutlich: Die Gruppe erwartet von jedem
ihrer Mitglieder nicht nur ein Verhalten, das sich an ihren Normen mehr oder
weniger genau orientiert; sie erzwingt es darüber hinaus geradezu, indem sie
erwartungsgemäßes Verhalten belohnt und erwartungswidriges Verhalten
unterbindet. Die soziale Kontrolle durch die Gruppierungen, in denen ein Mensch
heranwächst und in denen er sich aktuell befindet, hat eine fundamentale
pädagogische Bedeutung. Im Prinzip bleiben ihm vier Möglichkeiten, auf diese
sozialen Kontrollen zu reagieren: Er kann sich anpassen; er kann versuchen, die
Normen zu verändern; er kann als Abweichler in einer Randstellung der Gruppe
toleriert werden; er muss aus der Gruppe ausscheiden. Diese Möglichkeiten
lassen sich am Beispiel von Jugendlichen gut veranschaulichen, deren Verhalten
natürlich durch entwicklungspsychologische Faktoren beeinflusst wird, von denen
in diesem Zeitraum die Eltern-Kind-Beziehung auch mit abhängt. Der eine
Jugendliche übernimmt die Ansichten seiner Eltern und verhält sich in jeder
Beziehung so, wie diese es von ihm erwarten. Ein zweiter versucht mit Erfolg,
seine Eltern davon zu überzeugen, dass es ein alter Zopf und ihrer nicht würdig
sei, ihm zu verbieten, dass seine Freundin bei ihm im Zimmer übernachtet. Ein
dritter lebt, kleidet und frisiert sich wie ein Punker; seine Eltern können es
nicht fassen, ihn aber auch nicht anders beeinflussen; sie sehen sich machtlos,
gestehen ihm aber letzten Endes zu, auf seine Façon in ihrer Umgebung zu leben.
Der vierte verlässt wegen grundsätzlicher Meinungsverschiedenheiten die
elterliche Wohnung, oder er wird aus dieser hinausgeschmissen.

Soweit soziale Kontrolle von formellen Gruppierungen ausgeübt wird, kommt sie
nicht ohne äußere Sanktionen aus, um erwünschtes konformes Verhalten zu
erreichen. So gelang es erst durch Belegung mit einer Geldstrafe, die Pflicht
zum Angurten im Personenauto durchzusetzen und damit die Zahl der Unfallopfer
erheblich zu reduzieren. Die Schule bedient sich solcher äußerer Sanktionen,
bekannt als Schulstrafen, um das Einhalten der einschlägigen Bestimmungen durch
die Schüler durchzusetzen. Auch informelle Gruppierungen benutzen äußere
Sanktionen, \zB Klassenprügel für einen Mitschüler, haben aber in der Regel
wesentlich subtilere und damit umso wirksamere Methoden, soziale Kontrolle über
ihre Mitglieder auszuüben. Je wichtiger das Einhalten einer bestimmten Norm für
die Gruppe ist, desto stärker ist der Druck, den sie belohnend oder bestrafend
auf ihre Mitglieder ausübt. Da das Wertsystem der informellen Gruppe vom
einzelnen Individuum akzeptiert und auch internalisiert wird, empfindet es
einerseits diesen Druck viel weniger, als wenn er von einem formellen System
ausgeht, und sieht es andererseits in den notwendigen Verhaltensänderungen eher
eine Art eigenverantwortlicher Selbstkontrolle. So ergreifen beispielsweise
Trainer von Sportvereinen tief greifende und einschneidende Maßnahmen gegenüber
Spielern, die sich diese als Schüler von ihrem Sportlehrer nie gefallen lassen
würden. Pädagogisch gesehen wäre es demnach effektiver, sinnvoller und leichter
von den Betroffenen zu akzeptieren, wenn Werte und Normen vor allen Dingen über
informelle Gruppen tradiert oder auch geändert würden. Insofern ist die Rede
von der Familie als Keimzelle des Staates auch sicher heute noch richtig;
ebenso sind alle Ansätze zu begrüßen, die versuchen, formelle Gruppierungen mit
vorbestimmten Normen und äußeren Sanktionen durch informelle zu ersetzen, die
ihre Wertekonzept sich selbst suchen und dieses dann internalisiert auch in
ihre weitere Umgebung hineintragen.

\subsection{Interaktion}

\subsubsection{Kommunikation als Vorraussetzung}
\begin{itemize}
	\item[Definition IA:] Wechselseitige Beziehungen und Abhängigkeiten
		aufgrund von Gruppenaktivitäten, Gruppennormen, Sympathie und
		Rollenzuweisung.
	\item[verbale IA:] Inhalts- und Beziehungsaspekt (problemlos, da
		Gruppenmitglieder sich kennen)
	\item[nonverbale IA:] Mimik, Gestik, Körpersprache (kein Problem der
		Fehldeutung)
\end{itemize}

\subsubsection{Konformität}
Konformität der Einstellungen und Meinungen, bezogen auf Gruppenaktivität und
Gruppenziel, soll hergestellt werden.
\begin{itemize}
	\item Man gleicht sich dem Standard an (Sollwert) bezüglich der Leistung
	\item Vereinheitlichung von Meinungen (\zB Mathe-Unterricht in
		Parallelklassen bei demselben Lehrer)
	\item Als Gruppenmitglied sieht man sich in seinem Wesen und seinen
		Ansichten bestätigt
\end{itemize}
\emph{Folge:} Aus dem Maß an Konformität wächst Konstanz.

\subsubsection{Gruppenentscheidung}
Änderungen im normativen Verhalten und den Einstellungen von Gruppen sind
möglich:
\paragraph{4 Schritte:}
\begin{enumerate}
	\item Info aller Gruppenmitglieder
	\item Diskussion aller Vor- und Nachteile
	\item Lösungssuche
	\item Gemeinsames Ausprobieren
\end{enumerate}
Wichtig: \emph{Persönliche} Beteiligung

\paragraph{Themenzentrierte Interaktion}
Jede Gruppeninteraktion basiert auf 3 Ebenen: Sachebene, Wir-Ebene, Ich-Ebene
\begin{itemize}
	\item Ich = Persönlichkeit
	\item Wir = Gruppe
	\item Es = Thema
\end{itemize}
Eingeordnet in Zeit, Ort, historische und soziale Gegebenheiten.

\end{document}
